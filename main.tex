\documentclass{article}

\usepackage[utf8]{inputenc}
\usepackage[T1]{fontenc}
\usepackage{amsmath}
\usepackage{amsthm}
\usepackage{amssymb}
\usepackage{mathtools}
\usepackage{tikz}
\usepackage{setspace}
\usepackage{listings}
\usepackage{makeidx}
\usepackage{float}

\newcommand{\supp}{\mathrm{supp}}
\newcommand{\Z}{\mathbb{Z}}

\usetikzlibrary{arrows,chains,matrix,positioning,scopes}
\makeatletter
\tikzset{join/.code=\tikzset{after node path={%
\ifx\tikzchainprevious\pgfutil@empty\else(\tikzchainprevious)%
edge[every join]#1(\tikzchaincurrent)\fi}}}
\makeatother

\tikzset{>=stealth',every on chain/.append style={join},
         every join/.style={->}}
\tikzstyle{labeled}=[execute at begin node=$\scriptstyle,
   execute at end node=$]

\newtheorem{defi}{Definition}[section]
\newtheorem{lem}{Lemma}[section]
\newtheorem{conj}{Conjecture}[section]
\newtheorem{rem}{Remark}[section]
\newtheorem{prop}{Proposition}[section]
\newtheorem{thm}{Theorem}[section]
\newtheorem{cor}{Corollary}[section]
\newtheorem{expl}{Example}[section]

\usepackage{mathpazo}

\title{Expansion detected by cycles}
\author{Kai Renken \and Dmitry Kozlov}

\date{\today}

\begin{document}

\maketitle


\begin{abstract}

\end{abstract}

\section{On the hitting number of sets}

Let $S$ be a set and $\mathcal{F}\subseteq 2^S$ a family of subsets. Then
\[
\tau(\mathcal{F})\coloneqq\min\{|P|:P\subseteq S\text{, }P\cap F\neq\emptyset\text{, for all }F\in\mathcal{F}\}
\]
is called the \textbf{hitting number} of $\mathcal{F}$ and
\[
\Delta_{\mathcal{F}}\coloneqq\left\{S\subseteq\mathcal{F}:\bigcap\limits_{F\in S}F\neq\emptyset\right\}
\]
is called the \textbf{cut complex} of $\mathcal{F}$.\\
\\
Let now $X$ be a simplicial complex on the vertex set $V$ and $S\subseteq X$ a set of simplices of $X$, then we call $S$ a \textbf{covering} of $X$ if for every vertex $v\in V$ there exists a simplex $\sigma\in S$, such that $v\in\sigma$.\\
\\
The first obvious observation is that for every set $S$ and every family of subsets $\mathcal{F}\subseteq 2^S$ we have:
\[
\tau(\mathcal{F})=\min\{|S|:S\text{ is a covering of }\Delta_{\mathcal{F}}\}
\]

\section{The cut complex for families of cycles}

Denote the complete simplex on $n$ vertices by $\Delta^{[n]}$ and always work with coefficients in $\Z_2$. Let now $\varphi\in C^k(\Delta^{[n]})$ be a $k$-cochain. Then we can define the family
\[
T_{\varphi}\coloneqq\{\supp(\partial\sigma):\sigma\in\supp(\delta\varphi)\}
\]
where the sets are the supports of the boundaries of the simplices in the support of the coboundary of $\varphi$.\\
\\
Let us now study the structure of the cut complex of $T_{\varphi}$. First, we see that
\[
H_i(\Delta_{T_{\varphi}})=0
\]
holds for all $i\geq 2$, as follows:\\
If we have two $2$-simplices $\{v_1,v_2,v_3\}$ and $\{v_2,v_3,v_4\}$ in $\Delta_{T_{\varphi}}$ which share an edge $\{v_2,v_3\}$, then the $3$-simplex $\{v_1,v_2,v_3,v_4\}$ has to be contained in $\Delta_{T_{\varphi}}$ as well.
\begin{thebibliography}{9}

\end{thebibliography}

\end{document}


