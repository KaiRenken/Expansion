\documentclass{article}

\usepackage[utf8]{inputenc}
\usepackage[T1]{fontenc}
\usepackage{amsmath}
\usepackage{amsthm}
\usepackage{amssymb}
\usepackage{mathtools}
\usepackage{tikz}
\usepackage{setspace}
\usepackage{listings}
\usepackage{makeidx}
\usepackage{float}

\newcommand{\supp}{\mathrm{supp}}
\newcommand{\Z}{\mathbb{Z}}

\usetikzlibrary{arrows,chains,matrix,positioning,scopes}
\makeatletter
\tikzset{join/.code=\tikzset{after node path={%
\ifx\tikzchainprevious\pgfutil@empty\else(\tikzchainprevious)%
edge[every join]#1(\tikzchaincurrent)\fi}}}
\makeatother

\tikzset{>=stealth',every on chain/.append style={join},
         every join/.style={->}}
\tikzstyle{labeled}=[execute at begin node=$\scriptstyle,
   execute at end node=$]

\newtheorem{defi}{Definition}[section]
\newtheorem{lem}{Lemma}[section]
\newtheorem{conj}{Conjecture}[section]
\newtheorem{rem}{Remark}[section]
\newtheorem{prop}{Proposition}[section]
\newtheorem{thm}{Theorem}[section]
\newtheorem{cor}{Corollary}[section]
\newtheorem{expl}{Example}[section]

\usepackage{mathpazo}

\title{Expansion detected by cycles}
\author{Kai Renken \and Dmitry Kozlov}

\date{\today}

\begin{document}

\maketitle


\begin{abstract}

\end{abstract}

\section{Alternative Cheeger constant of the simplex}

\begin{thm}
Let $n$ not be a power of $2$, then we have:
\[
\gamma_1(\Delta^{[n]})=\frac{n}{3}
\] 
\begin{proof}
Since $n$ is not a power of $2$ we can write it as $n=c(2t+1)$. Now consider the staircase graph $G_n(\lambda)$ given by the partition $\lambda=c\cdot\text{cor}(t)$. Since $G_n(\lambda)$ is bipartite we can partition the vertices of $G_n(\lambda)$ as $[n]=A\cup B\cup C$, with $A=\{v_1,\ldots,v_{ct}\}$, $B=\{w_1,\ldots,w_{ct}\}$ and $C=\{x_1,\ldots,x_c\}$, such that $C$ is the set of all isolated vertices and all edges of $G_n(\lambda)$ are contained in $E_{G_n(\lambda)}(A,B)$.\\
Construct a family of edge-disjoint cycles of the vertex set $[n]$ as follows:\\
For all edges $(v_i,w_j)$ satisfying $i+j\leq ct$, such that $(v_{ct-j+1},w_{ct-i+1})$ is not an edge in $G_n(\lambda)$ consider the cycle
\[
C_{ij}\coloneqq\{(v_i,w_j),(v_{ct-j+1},w_{ct-i+1}),(v_i,v_{ct-j+1}),(w_j,w_{ct-i+1})\}
\]
For all edges $e_{ij}=(v_i,w_j)$ satisfying $i+j\leq ct+1$, such that $e'_{ij}=(v_{ct-j+1},w_{ct-i+1})$ is also an edge in $G_n(\lambda)$ (for $i+j=ct+1$ they are equal), the set
\[
D\coloneqq\{e_{ij}, e'_{ij}:i+j\leq ct+1\text{, }e_{ij}\text{ and }e'_{ij}\text{ are edges in }G_n(\lambda)\}
\] can be partitioned into $t$ sets $B_1,\ldots,B_t$, each containing $c^2$ edges:
\[
B_k\coloneqq\{(v_i,w_j):(k-1)c+1\leq i\leq kc\text{, }c(t-k)+1\leq j\leq c(t-k+1)\}
\]
Now, each vertex from $A\cup B$ is only contained in edges from exactly one of the sets $B_k$. This means that for any $l=1,\ldots,c$ and any pair of edges $(v_i,w_j)\in B_{k_1}$ and $(v_{i'},w_{j'})\in B_{k_2}$ ($k_1\neq k_2$) the cycles $\{(v_i,w_j),(v_i,x_l),(w_j,x_l)\}$ and $\{(v_{i'},w_{j'}),(v_{i'},x_l),(w_{j'},x_l)\}$ are edge-disjoint. Furthermore, each set $B_k$ itself is a complete balanced bipartite graph (i.e. a graph in which each of the $c$ vertices from $A$ is adjecent to each of the $c$ vertices from $B$) so we can partition it into $c$ sets $B_k^1,\ldots,B_k^c$, such that all edges in $B_k^l$ are disjoint, for every $l=1,\ldots,c$. Thus, the cycles $\{(v_i,w_j),(v_i,x_l),(w_j,x_l)\}$ are edge-disjoint for all $(v_i,w_j)\in B_k^l$. The family of all these cycles united with the cycles $C_{ij}$ we defined before gives a family of edge-disjoint cycles, such that every edge of $G_n(\lambda)$ is contained in exactly one cycle and every cycle containes exactly one of the edges from $G_n(\lambda)$. Hence, the hitting number of this 
\end{proof}
\end{thm}

\begin{thm}
Let $k+2$ devide $n$, then we have:
\[
\gamma_k(\Delta^{[n]})=\frac{n}{k+2}
\]
\begin{proof}
Follows immediately from \cite{1}
\end{proof}
\end{thm}



\begin{thebibliography}{9}

\bibitem{1} Dmitry N. Kozlov, Roy Meshulam; Quantitative aspects of acyclicity

\end{thebibliography}

\end{document}


